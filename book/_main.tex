% Options for packages loaded elsewhere
\PassOptionsToPackage{unicode}{hyperref}
\PassOptionsToPackage{hyphens}{url}
%
\documentclass[
]{book}
\usepackage{amsmath,amssymb}
\usepackage{iftex}
\ifPDFTeX
  \usepackage[T1]{fontenc}
  \usepackage[utf8]{inputenc}
  \usepackage{textcomp} % provide euro and other symbols
\else % if luatex or xetex
  \usepackage{unicode-math} % this also loads fontspec
  \defaultfontfeatures{Scale=MatchLowercase}
  \defaultfontfeatures[\rmfamily]{Ligatures=TeX,Scale=1}
\fi
\usepackage{lmodern}
\ifPDFTeX\else
  % xetex/luatex font selection
\fi
% Use upquote if available, for straight quotes in verbatim environments
\IfFileExists{upquote.sty}{\usepackage{upquote}}{}
\IfFileExists{microtype.sty}{% use microtype if available
  \usepackage[]{microtype}
  \UseMicrotypeSet[protrusion]{basicmath} % disable protrusion for tt fonts
}{}
\makeatletter
\@ifundefined{KOMAClassName}{% if non-KOMA class
  \IfFileExists{parskip.sty}{%
    \usepackage{parskip}
  }{% else
    \setlength{\parindent}{0pt}
    \setlength{\parskip}{6pt plus 2pt minus 1pt}}
}{% if KOMA class
  \KOMAoptions{parskip=half}}
\makeatother
\usepackage{xcolor}
\usepackage{color}
\usepackage{fancyvrb}
\newcommand{\VerbBar}{|}
\newcommand{\VERB}{\Verb[commandchars=\\\{\}]}
\DefineVerbatimEnvironment{Highlighting}{Verbatim}{commandchars=\\\{\}}
% Add ',fontsize=\small' for more characters per line
\usepackage{framed}
\definecolor{shadecolor}{RGB}{248,248,248}
\newenvironment{Shaded}{\begin{snugshade}}{\end{snugshade}}
\newcommand{\AlertTok}[1]{\textcolor[rgb]{0.94,0.16,0.16}{#1}}
\newcommand{\AnnotationTok}[1]{\textcolor[rgb]{0.56,0.35,0.01}{\textbf{\textit{#1}}}}
\newcommand{\AttributeTok}[1]{\textcolor[rgb]{0.13,0.29,0.53}{#1}}
\newcommand{\BaseNTok}[1]{\textcolor[rgb]{0.00,0.00,0.81}{#1}}
\newcommand{\BuiltInTok}[1]{#1}
\newcommand{\CharTok}[1]{\textcolor[rgb]{0.31,0.60,0.02}{#1}}
\newcommand{\CommentTok}[1]{\textcolor[rgb]{0.56,0.35,0.01}{\textit{#1}}}
\newcommand{\CommentVarTok}[1]{\textcolor[rgb]{0.56,0.35,0.01}{\textbf{\textit{#1}}}}
\newcommand{\ConstantTok}[1]{\textcolor[rgb]{0.56,0.35,0.01}{#1}}
\newcommand{\ControlFlowTok}[1]{\textcolor[rgb]{0.13,0.29,0.53}{\textbf{#1}}}
\newcommand{\DataTypeTok}[1]{\textcolor[rgb]{0.13,0.29,0.53}{#1}}
\newcommand{\DecValTok}[1]{\textcolor[rgb]{0.00,0.00,0.81}{#1}}
\newcommand{\DocumentationTok}[1]{\textcolor[rgb]{0.56,0.35,0.01}{\textbf{\textit{#1}}}}
\newcommand{\ErrorTok}[1]{\textcolor[rgb]{0.64,0.00,0.00}{\textbf{#1}}}
\newcommand{\ExtensionTok}[1]{#1}
\newcommand{\FloatTok}[1]{\textcolor[rgb]{0.00,0.00,0.81}{#1}}
\newcommand{\FunctionTok}[1]{\textcolor[rgb]{0.13,0.29,0.53}{\textbf{#1}}}
\newcommand{\ImportTok}[1]{#1}
\newcommand{\InformationTok}[1]{\textcolor[rgb]{0.56,0.35,0.01}{\textbf{\textit{#1}}}}
\newcommand{\KeywordTok}[1]{\textcolor[rgb]{0.13,0.29,0.53}{\textbf{#1}}}
\newcommand{\NormalTok}[1]{#1}
\newcommand{\OperatorTok}[1]{\textcolor[rgb]{0.81,0.36,0.00}{\textbf{#1}}}
\newcommand{\OtherTok}[1]{\textcolor[rgb]{0.56,0.35,0.01}{#1}}
\newcommand{\PreprocessorTok}[1]{\textcolor[rgb]{0.56,0.35,0.01}{\textit{#1}}}
\newcommand{\RegionMarkerTok}[1]{#1}
\newcommand{\SpecialCharTok}[1]{\textcolor[rgb]{0.81,0.36,0.00}{\textbf{#1}}}
\newcommand{\SpecialStringTok}[1]{\textcolor[rgb]{0.31,0.60,0.02}{#1}}
\newcommand{\StringTok}[1]{\textcolor[rgb]{0.31,0.60,0.02}{#1}}
\newcommand{\VariableTok}[1]{\textcolor[rgb]{0.00,0.00,0.00}{#1}}
\newcommand{\VerbatimStringTok}[1]{\textcolor[rgb]{0.31,0.60,0.02}{#1}}
\newcommand{\WarningTok}[1]{\textcolor[rgb]{0.56,0.35,0.01}{\textbf{\textit{#1}}}}
\usepackage{longtable,booktabs,array}
\usepackage{calc} % for calculating minipage widths
% Correct order of tables after \paragraph or \subparagraph
\usepackage{etoolbox}
\makeatletter
\patchcmd\longtable{\par}{\if@noskipsec\mbox{}\fi\par}{}{}
\makeatother
% Allow footnotes in longtable head/foot
\IfFileExists{footnotehyper.sty}{\usepackage{footnotehyper}}{\usepackage{footnote}}
\makesavenoteenv{longtable}
\usepackage{graphicx}
\makeatletter
\def\maxwidth{\ifdim\Gin@nat@width>\linewidth\linewidth\else\Gin@nat@width\fi}
\def\maxheight{\ifdim\Gin@nat@height>\textheight\textheight\else\Gin@nat@height\fi}
\makeatother
% Scale images if necessary, so that they will not overflow the page
% margins by default, and it is still possible to overwrite the defaults
% using explicit options in \includegraphics[width, height, ...]{}
\setkeys{Gin}{width=\maxwidth,height=\maxheight,keepaspectratio}
% Set default figure placement to htbp
\makeatletter
\def\fps@figure{htbp}
\makeatother
\setlength{\emergencystretch}{3em} % prevent overfull lines
\providecommand{\tightlist}{%
  \setlength{\itemsep}{0pt}\setlength{\parskip}{0pt}}
\setcounter{secnumdepth}{5}
\usepackage{booktabs}
\ifLuaTeX
  \usepackage{selnolig}  % disable illegal ligatures
\fi
\usepackage[]{natbib}
\bibliographystyle{plainnat}
\IfFileExists{bookmark.sty}{\usepackage{bookmark}}{\usepackage{hyperref}}
\IfFileExists{xurl.sty}{\usepackage{xurl}}{} % add URL line breaks if available
\urlstyle{same}
\hypersetup{
  pdftitle={Financial Econometrics - Tutorials},
  pdfauthor={Alessandro Ciancetta},
  hidelinks,
  pdfcreator={LaTeX via pandoc}}

\title{Financial Econometrics - Tutorials}
\author{Alessandro Ciancetta}
\date{2024-01-16}

\usepackage{amsthm}
\newtheorem{theorem}{Theorem}[chapter]
\newtheorem{lemma}{Lemma}[chapter]
\newtheorem{corollary}{Corollary}[chapter]
\newtheorem{proposition}{Proposition}[chapter]
\newtheorem{conjecture}{Conjecture}[chapter]
\theoremstyle{definition}
\newtheorem{definition}{Definition}[chapter]
\theoremstyle{definition}
\newtheorem{example}{Example}[chapter]
\theoremstyle{definition}
\newtheorem{exercise}{Exercise}[chapter]
\theoremstyle{definition}
\newtheorem{hypothesis}{Hypothesis}[chapter]
\theoremstyle{remark}
\newtheorem*{remark}{Remark}
\newtheorem*{solution}{Solution}
\begin{document}
\maketitle

{
\setcounter{tocdepth}{1}
\tableofcontents
}
\hypertarget{about}{%
\chapter{About}\label{about}}

TA materials for the Financial Econometrics course held by Prof.~Christian Brownlees at the Barcelona School of Economics.

\hypertarget{session01}{%
\chapter{Session 1}\label{session01}}

\hypertarget{stochastic-processes-and-dependence}{%
\section{Stochastic processes and dependence}\label{stochastic-processes-and-dependence}}

Stochastic processes allow for dependence in consecutive random variables \(\{\dots, Y_{-2}, Y_{-1}, Y_{0}, Y_{1}, Y_{2} \dots\}\). However, in practice, when we observe an empirical time series we are considering \emph{one, truncated} realization of the stochastic process \(\{y_1, y_2, \dots, y_T\}\). As it is easy to imagine, this can cause some issues in studying the properties of an empirical time series. First, because we can only study the \emph{finite} dimensional distribution of the process. Second, because our task is to learn something about the process using a single realization.

To overcome this limitations we need assumptions. In particular, two common assumptions in time series analysis are, loosely speaking:
* stationarity: the observed values in the sequence come from the same distribution, so that it is possible to learn from the past observations and to generalize the results to the entire, infinite stochastic process
* ergodicity: values observed far away in time can be considered as independent, so that the sufficiently long series can be considered as a representative sample of the underlying distribution

Under these (or similar) assumptions, we can use the observations within a single observed time series to learn the parameters of our models.

\hypertarget{stationarity}{%
\subsection*{Stationarity}\label{stationarity}}
\addcontentsline{toc}{subsection}{Stationarity}

\begin{Shaded}
\begin{Highlighting}[]
\DocumentationTok{\#\# Example 1: non{-}stationary series}
\CommentTok{\# simulation}
\NormalTok{t\_max }\OtherTok{\textless{}{-}} \DecValTok{500}
\NormalTok{y }\OtherTok{\textless{}{-}} \FunctionTok{rep}\NormalTok{(}\ConstantTok{NA}\NormalTok{, t\_max)}
\ControlFlowTok{for}\NormalTok{ (t }\ControlFlowTok{in} \DecValTok{1}\SpecialCharTok{:}\NormalTok{t\_max) \{}
  \ControlFlowTok{if}\NormalTok{ (t}\SpecialCharTok{\textless{}=}\DecValTok{100}\NormalTok{) \{}
\NormalTok{    y[t] }\OtherTok{\textless{}{-}} \FunctionTok{rnorm}\NormalTok{(}\DecValTok{1}\NormalTok{, }\AttributeTok{mean =} \DecValTok{0}\NormalTok{, }\AttributeTok{sd =} \FloatTok{0.1}\NormalTok{)}
\NormalTok{  \}}
  \ControlFlowTok{if}\NormalTok{ (t}\SpecialCharTok{\textgreater{}}\DecValTok{100} \SpecialCharTok{\&}\NormalTok{ t}\SpecialCharTok{\textless{}=}\DecValTok{250}\NormalTok{) \{}
\NormalTok{    y[t] }\OtherTok{\textless{}{-}} \FunctionTok{rnorm}\NormalTok{(}\DecValTok{1}\NormalTok{, }\AttributeTok{mean =} \FloatTok{0.5}\NormalTok{, }\AttributeTok{sd =} \FloatTok{0.1}\NormalTok{)}
\NormalTok{  \}}
  \ControlFlowTok{if}\NormalTok{ (t}\SpecialCharTok{\textgreater{}}\DecValTok{250} \SpecialCharTok{\&}\NormalTok{ t}\SpecialCharTok{\textless{}=}\DecValTok{400}\NormalTok{) \{}
\NormalTok{    y[t] }\OtherTok{\textless{}{-}} \FunctionTok{rnorm}\NormalTok{(}\DecValTok{1}\NormalTok{, }\AttributeTok{mean =} \DecValTok{0}\NormalTok{, }\AttributeTok{sd =} \FloatTok{0.2}\NormalTok{)}
\NormalTok{  \}}
  \ControlFlowTok{if}\NormalTok{ (t}\SpecialCharTok{\textgreater{}}\DecValTok{400} \SpecialCharTok{\&}\NormalTok{ t}\SpecialCharTok{\textless{}=}\NormalTok{t\_max) \{}
\NormalTok{    y[t] }\OtherTok{\textless{}{-}} \FunctionTok{rnorm}\NormalTok{(}\DecValTok{1}\NormalTok{, }\AttributeTok{mean =} \SpecialCharTok{{-}}\FloatTok{0.2}\NormalTok{, }\AttributeTok{sd =} \FloatTok{0.05}\NormalTok{)}
\NormalTok{  \}}
\NormalTok{\}}
\CommentTok{\# plot}
\FunctionTok{plot.ts}\NormalTok{(y, }\AttributeTok{main =} \StringTok{"Non{-}stationary series"}\NormalTok{)}
\end{Highlighting}
\end{Shaded}

\includegraphics{_main_files/figure-latex/unnamed-chunk-1-1.pdf}

\hypertarget{ergodicity}{%
\subsection*{Ergodicity}\label{ergodicity}}
\addcontentsline{toc}{subsection}{Ergodicity}

This examples shows how the assumptions of stationarity and ergodicity allow to learn the parameters of the distribution of the stochastic process from an empirical time series.

Let \(\{z_t\}_{t = 1}^{T} \stackrel{iid}{\sim} \mathcal{N}(0, 1)\)

\[
\begin{aligned}
y_t &= U_0 + 0.25z_t, \quad U_0 \sim \mathcal{N}(0, 10) \\
x_t &= z_t + z_{t-1}
\end{aligned}
\]
We have:
\[
\mathbb{E}[y_t] = \mathbb{E}[x_t] = 0
\]
\[
\begin{aligned}
\text{cov}(y_t, y_{t-1}) &= 
\begin{cases}
1+0.25^2 & h = 0 \\
1 & h \neq 0
\end{cases} \\[1em]
\text{cov}(x_t, x_{t-1}) &= 
\begin{cases}
2 & h = 0 \\
1 & h = 1 \\
0 & h \geq 2
\end{cases}
\end{aligned}
\]

\begin{Shaded}
\begin{Highlighting}[]
\DocumentationTok{\#\# Example 2: weak dependence}
\CommentTok{\# initialize object to store result}
\NormalTok{nsim }\OtherTok{\textless{}{-}} \DecValTok{3}
\NormalTok{y\_list }\OtherTok{\textless{}{-}} \FunctionTok{matrix}\NormalTok{(}\FunctionTok{rep}\NormalTok{(}\ConstantTok{NA}\NormalTok{, nsim}\SpecialCharTok{*}\NormalTok{t\_max), }\AttributeTok{nrow =}\NormalTok{ t\_max, }\AttributeTok{ncol =}\NormalTok{ nsim)}
\CommentTok{\# simulation}
\ControlFlowTok{for}\NormalTok{ (sim }\ControlFlowTok{in} \DecValTok{1}\SpecialCharTok{:}\NormalTok{nsim) \{}
  \FunctionTok{set.seed}\NormalTok{(sim}\SpecialCharTok{+}\DecValTok{123}\NormalTok{)}
\NormalTok{  U0 }\OtherTok{\textless{}{-}} \FunctionTok{rnorm}\NormalTok{(}\DecValTok{1}\NormalTok{, }\AttributeTok{sd =} \DecValTok{10}\NormalTok{)}
\NormalTok{  z  }\OtherTok{\textless{}{-}} \FunctionTok{rnorm}\NormalTok{(t\_max)}
\NormalTok{  y\_list[,sim] }\OtherTok{\textless{}{-}}\NormalTok{ U0 }\SpecialCharTok{+}\NormalTok{ z}
\NormalTok{\}}
\CommentTok{\# plot}
\FunctionTok{plot.ts}\NormalTok{(y\_list[,}\DecValTok{1}\NormalTok{], }\AttributeTok{ylim =} \FunctionTok{c}\NormalTok{(}\FunctionTok{min}\NormalTok{(y\_list), }\FunctionTok{max}\NormalTok{(y\_list)), }\AttributeTok{main =} \StringTok{"Realizations of non{-}ergodic series"}\NormalTok{, }\AttributeTok{ylab =} \StringTok{"y"}\NormalTok{)}
\FunctionTok{lines}\NormalTok{(y\_list[,}\DecValTok{2}\NormalTok{], }\AttributeTok{col =} \StringTok{"steelblue"}\NormalTok{)}
\FunctionTok{lines}\NormalTok{(y\_list[,}\DecValTok{3}\NormalTok{], }\AttributeTok{col =} \StringTok{"tomato"}\NormalTok{)}
\FunctionTok{abline}\NormalTok{(}\AttributeTok{h =} \DecValTok{0}\NormalTok{, }\AttributeTok{col =} \StringTok{"black"}\NormalTok{, }\AttributeTok{lwd =} \DecValTok{2}\NormalTok{, }\AttributeTok{lty =} \DecValTok{2}\NormalTok{)}
\end{Highlighting}
\end{Shaded}

\includegraphics{_main_files/figure-latex/unnamed-chunk-2-1.pdf}

\begin{Shaded}
\begin{Highlighting}[]
\CommentTok{\# initialize object to store result}
\NormalTok{nsim }\OtherTok{\textless{}{-}} \DecValTok{10000}
\NormalTok{x\_list }\OtherTok{\textless{}{-}} \FunctionTok{matrix}\NormalTok{(}\FunctionTok{rep}\NormalTok{(}\ConstantTok{NA}\NormalTok{, nsim}\SpecialCharTok{*}\NormalTok{t\_max), }\AttributeTok{nrow =}\NormalTok{ t\_max, }\AttributeTok{ncol =}\NormalTok{ nsim)}
\CommentTok{\# simulation}
\ControlFlowTok{for}\NormalTok{ (sim }\ControlFlowTok{in} \DecValTok{1}\SpecialCharTok{:}\NormalTok{nsim) \{}
  \FunctionTok{set.seed}\NormalTok{(sim}\SpecialCharTok{+}\DecValTok{123}\NormalTok{)}
\NormalTok{  z  }\OtherTok{\textless{}{-}} \FunctionTok{rnorm}\NormalTok{(t\_max}\SpecialCharTok{+}\DecValTok{1}\NormalTok{)}
\NormalTok{  x\_list[,sim] }\OtherTok{\textless{}{-}}\NormalTok{ z[}\DecValTok{2}\SpecialCharTok{:}\NormalTok{(t\_max}\SpecialCharTok{+}\DecValTok{1}\NormalTok{)] }\SpecialCharTok{+}\NormalTok{ z[}\DecValTok{1}\SpecialCharTok{:}\NormalTok{t\_max]}
\NormalTok{\}}
\CommentTok{\# plot}
\FunctionTok{plot.ts}\NormalTok{(x\_list[,}\DecValTok{1}\NormalTok{], }\AttributeTok{ylim =} \FunctionTok{c}\NormalTok{(}\FunctionTok{min}\NormalTok{(x\_list), }\FunctionTok{max}\NormalTok{(x\_list)), }\AttributeTok{main =} \StringTok{"Realizations of ergodic series"}\NormalTok{, }\AttributeTok{ylab =} \StringTok{"y"}\NormalTok{)}
\FunctionTok{lines}\NormalTok{(x\_list[,}\DecValTok{2}\NormalTok{], }\AttributeTok{col =} \StringTok{"steelblue"}\NormalTok{)}
\FunctionTok{lines}\NormalTok{(x\_list[,}\DecValTok{3}\NormalTok{], }\AttributeTok{col =} \StringTok{"tomato"}\NormalTok{)}
\FunctionTok{abline}\NormalTok{(}\AttributeTok{h =} \DecValTok{0}\NormalTok{, }\AttributeTok{col =} \StringTok{"black"}\NormalTok{, }\AttributeTok{lwd =} \DecValTok{2}\NormalTok{, }\AttributeTok{lty =} \DecValTok{2}\NormalTok{)}
\end{Highlighting}
\end{Shaded}

\includegraphics{_main_files/figure-latex/unnamed-chunk-3-1.pdf}

\hypertarget{asymptotic-results}{%
\section{Asymptotic results}\label{asymptotic-results}}

The problem with the series \(\{y_t\}\) in the previous example is that the autocovariance function does not decay. Loosely speaking, the series gets stuck in the trajectory given by the initial draw of \(U_0\) and does not revert to the true mean of the stochastic process. The main consequence is that we cannot learn the mean of the process by taking the average of the observations in a given time series.

The fact that some conditions about the decay of the autocovariance are required to recover the mean of the process is true in general. For example, the Law of Large Numbers (LLN) guarantees that the sample mean converges in probability to the true mean under the assumption that the autocovariances are absolutely summable:

\[
\sum_{k=0}^\infty |\gamma_k| < \infty
\]

Notice that in the case of \(\{y_t\}\) above instead \(\sum_{k=0}^\infty |\gamma_k| = \infty\). On the contrary, \(\{x_t\}\) satisfies both the conditions of the LLN and of the Central Limit Theorem (CLT). The condition for the latter is that \(\{\phi_k\}_{k=0}^\infty\) is absolutely summable in \(x_t = \mu + \sum_{k=0}^\infty\phi_k z_{t-k} = z_t + z_{t-1}\), which is trivially verified. Therefore, since \(\mu_x = 0\),

\[
\sqrt{T} \ \bar{x}_T  \ \xrightarrow{d} \ \mathcal{N}(0, \sigma^2_{LR}),
\]

with \(\sigma^2_{LR} = \sum_{k=-\infty}^\infty \gamma_k = \text{Var}(x_t) + 2\sum_{k=1}^\infty \gamma_k\).

\begin{Shaded}
\begin{Highlighting}[]
\DocumentationTok{\#\# Example 3: central limit theorem}
\NormalTok{x\_empirical }\OtherTok{\textless{}{-}}\NormalTok{ x\_list[,}\DecValTok{9}\NormalTok{]}
\NormalTok{x\_means }\OtherTok{\textless{}{-}} \FunctionTok{colMeans}\NormalTok{(x\_list)}
\NormalTok{x\_theory\_variance }\OtherTok{\textless{}{-}} \DecValTok{4}\SpecialCharTok{/}\NormalTok{t\_max }

\FunctionTok{rbind}\NormalTok{(}\AttributeTok{empirical\_variance =} \FunctionTok{var}\NormalTok{(x\_empirical) }\SpecialCharTok{+} \DecValTok{2}\SpecialCharTok{*}\NormalTok{(}\FunctionTok{cov}\NormalTok{(x\_empirical[}\DecValTok{1}\SpecialCharTok{:}\NormalTok{(t\_max}\DecValTok{{-}1}\NormalTok{)], x\_empirical[}\DecValTok{2}\SpecialCharTok{:}\NormalTok{t\_max])),}
      \AttributeTok{simulated\_variance =} \FunctionTok{var}\NormalTok{(x\_means}\SpecialCharTok{*}\FunctionTok{sqrt}\NormalTok{(t\_max)),}
      \AttributeTok{theoretical\_variance =}\NormalTok{ x\_theory\_variance}\SpecialCharTok{*}\NormalTok{t\_max)}
\end{Highlighting}
\end{Shaded}

\begin{verbatim}
##                          [,1]
## empirical_variance   4.174614
## simulated_variance   4.050667
## theoretical_variance 4.000000
\end{verbatim}

\begin{Shaded}
\begin{Highlighting}[]
\DocumentationTok{\#\# Plot empirical distribution VS. theoretical distribution}
\FunctionTok{hist}\NormalTok{(x\_means, }\AttributeTok{breaks =} \DecValTok{20}\NormalTok{, }\AttributeTok{freq =} \ConstantTok{FALSE}\NormalTok{, }
     \AttributeTok{main =} \StringTok{"Distribution of the simulated means"}\NormalTok{, }
     \AttributeTok{xlab =} \StringTok{"Simulated means"}\NormalTok{)}
\FunctionTok{lines}\NormalTok{(}\FunctionTok{density}\NormalTok{(x\_means), }\AttributeTok{lwd =} \DecValTok{4}\NormalTok{, }\AttributeTok{col =} \StringTok{"tomato"}\NormalTok{)}
\FunctionTok{lines}\NormalTok{(}\FunctionTok{density}\NormalTok{(}\FunctionTok{rnorm}\NormalTok{(}\FloatTok{1e6}\NormalTok{, }\AttributeTok{mean =} \DecValTok{0}\NormalTok{, }\AttributeTok{sd =} \FunctionTok{sqrt}\NormalTok{(x\_theory\_variance))), }\AttributeTok{lwd =} \DecValTok{4}\NormalTok{, }\AttributeTok{col =} \StringTok{"darkblue"}\NormalTok{)}
\end{Highlighting}
\end{Shaded}

\includegraphics{_main_files/figure-latex/unnamed-chunk-5-1.pdf}

\hypertarget{empirical-moments-and-autocorrelograms}{%
\section{Empirical moments and autocorrelograms}\label{empirical-moments-and-autocorrelograms}}

For this example we use the \includegraphics{U.S. GDP data}.

\begin{Shaded}
\begin{Highlighting}[]
\NormalTok{x }\OtherTok{\textless{}{-}} \FunctionTok{read.csv}\NormalTok{(}\StringTok{"../us{-}gdp.csv"}\NormalTok{)[,}\DecValTok{2}\NormalTok{]}
\NormalTok{x }\OtherTok{\textless{}{-}} \FunctionTok{ts}\NormalTok{(x, }\AttributeTok{start =} \FunctionTok{c}\NormalTok{(}\DecValTok{1947}\NormalTok{, }\DecValTok{1}\NormalTok{), }\AttributeTok{frequency =} \DecValTok{4}\NormalTok{)}
\NormalTok{t\_max }\OtherTok{\textless{}{-}} \FunctionTok{length}\NormalTok{(x)}
\FunctionTok{plot.ts}\NormalTok{(x, }\AttributeTok{main =} \StringTok{"U.S. GDP"}\NormalTok{, }\AttributeTok{ylab =} \StringTok{"Billions of dollars"}\NormalTok{)}
\end{Highlighting}
\end{Shaded}

\includegraphics{_main_files/figure-latex/unnamed-chunk-6-1.pdf}
We consider the annualized quarterly growth rates:

\begin{Shaded}
\begin{Highlighting}[]
\CommentTok{\# xgrowth \textless{}{-} ( (x[2:t\_max]/x[1:(t\_max{-}1)])\^{}4 {-} 1 )*100}
\NormalTok{xgrowth }\OtherTok{\textless{}{-}} \DecValTok{4} \SpecialCharTok{*} \FunctionTok{diff}\NormalTok{( }\FunctionTok{log}\NormalTok{(x) ) }\SpecialCharTok{*} \DecValTok{100}
\NormalTok{xgrowth }\OtherTok{\textless{}{-}} \FunctionTok{ts}\NormalTok{(xgrowth, }\AttributeTok{start =} \FunctionTok{c}\NormalTok{(}\DecValTok{1947}\NormalTok{, }\DecValTok{2}\NormalTok{), }\AttributeTok{frequency =} \DecValTok{4}\NormalTok{)}
\FunctionTok{plot.ts}\NormalTok{(xgrowth, }\AttributeTok{main =} \StringTok{"Annualized GDP growth"}\NormalTok{)}
\end{Highlighting}
\end{Shaded}

\includegraphics{_main_files/figure-latex/unnamed-chunk-7-1.pdf}

\begin{Shaded}
\begin{Highlighting}[]
\DocumentationTok{\#\# Example 4: empirical moments of the GDP growth}
\FunctionTok{library}\NormalTok{(moments)}
\FunctionTok{rbind}\NormalTok{(}
  \AttributeTok{mean =} \FunctionTok{mean}\NormalTok{(xgrowth),}
  \AttributeTok{variance =} \FunctionTok{var}\NormalTok{(xgrowth),}
  \AttributeTok{skewness =} \FunctionTok{skewness}\NormalTok{(xgrowth),}
  \AttributeTok{kurtosis =} \FunctionTok{kurtosis}\NormalTok{(xgrowth)}
\NormalTok{)}
\end{Highlighting}
\end{Shaded}

\begin{verbatim}
##                [,1]
## mean      6.1858847
## variance 26.5117326
## skewness -0.9793909
## kurtosis 17.9597257
\end{verbatim}

\begin{Shaded}
\begin{Highlighting}[]
\DocumentationTok{\#\# Autocovariance function}
\NormalTok{gamma }\OtherTok{\textless{}{-}} \ControlFlowTok{function}\NormalTok{(x, k) \{}
\NormalTok{  k }\OtherTok{\textless{}{-}} \FunctionTok{abs}\NormalTok{(k)}
\NormalTok{  t\_max }\OtherTok{\textless{}{-}} \FunctionTok{length}\NormalTok{(x)}
  \CommentTok{\# (t\_max{-}k)/t\_max * cov(x[1:(length(x){-}k)], x[(k+1):length(x)]) \# for compatibility with acf()}
  \FunctionTok{cov}\NormalTok{(x[}\DecValTok{1}\SpecialCharTok{:}\NormalTok{(t\_max}\SpecialCharTok{{-}}\NormalTok{k)], x[(k}\SpecialCharTok{+}\DecValTok{1}\NormalTok{)}\SpecialCharTok{:}\NormalTok{t\_max])}
\NormalTok{\}}

\DocumentationTok{\#\# Autocorrelation}
\NormalTok{rho }\OtherTok{\textless{}{-}} \ControlFlowTok{function}\NormalTok{(x, k) \{}\FunctionTok{gamma}\NormalTok{(x, k) }\SpecialCharTok{/} \FunctionTok{gamma}\NormalTok{(x, }\DecValTok{0}\NormalTok{)\}}

\DocumentationTok{\#\# autocorrelation at different lags}
\FunctionTok{sapply}\NormalTok{(}\DecValTok{0}\SpecialCharTok{:}\DecValTok{12}\NormalTok{, rho, }\AttributeTok{x =}\NormalTok{ xgrowth)}
\end{Highlighting}
\end{Shaded}

\begin{verbatim}
##  [1]  1.000000000  0.262228567  0.254514380  0.097922439  0.030475252
##  [6] -0.016390183 -0.003186283  0.054980002  0.062358818  0.146571003
## [11]  0.169620140  0.143289793  0.096325249
\end{verbatim}

Under the null hypothesis \(H_0: \rho = 0\), the sample autocorrelation is distributed as
\[
\sqrt{T} \ \hat{\rho} \xrightarrow{d} \mathcal{N}(0, 1)
\]

This means that the asymptotic variance of the estimator under the null is \(1/T\). The plot reports the 95\% confidence interval obtained as \(\left(0 \pm \frac{z_{0.975}}{\sqrt{T}}\right)\).

\begin{Shaded}
\begin{Highlighting}[]
\DocumentationTok{\#\# using built{-}in function for the autocorrelogram}
\FunctionTok{acf}\NormalTok{(xgrowth, }\AttributeTok{lag.max =} \DecValTok{16}\NormalTok{)}
\end{Highlighting}
\end{Shaded}

\includegraphics{_main_files/figure-latex/unnamed-chunk-10-1.pdf}

\begin{Shaded}
\begin{Highlighting}[]
\DocumentationTok{\#\# partial autocorrelation function}
\FunctionTok{pacf}\NormalTok{(xgrowth, }\AttributeTok{lag.max =} \DecValTok{16}\NormalTok{)}
\end{Highlighting}
\end{Shaded}

\includegraphics{_main_files/figure-latex/unnamed-chunk-11-1.pdf}

\hypertarget{hypothesis-testing}{%
\section{Hypothesis testing}\label{hypothesis-testing}}

In this section we use three testing procedures on the GDP growth data: the Augmented Dickey-Fuller test for stationarity, the Jarque-Bera test for normality and the t-test for the mean of a process.

\begin{Shaded}
\begin{Highlighting}[]
\DocumentationTok{\#\# Example 5: tests on GDP growth rates }
\FunctionTok{library}\NormalTok{(tseries)}

\DocumentationTok{\#\# stationarity: augmented Dickey{-}Fuller}
\FunctionTok{adf.test}\NormalTok{(x)}
\end{Highlighting}
\end{Shaded}

\begin{verbatim}
## 
##  Augmented Dickey-Fuller Test
## 
## data:  x
## Dickey-Fuller = 2.4611, Lag order = 6, p-value = 0.99
## alternative hypothesis: stationary
\end{verbatim}

\begin{Shaded}
\begin{Highlighting}[]
\FunctionTok{adf.test}\NormalTok{(xgrowth)}
\end{Highlighting}
\end{Shaded}

\begin{verbatim}
## 
##  Augmented Dickey-Fuller Test
## 
## data:  xgrowth
## Dickey-Fuller = -5.9064, Lag order = 6, p-value = 0.01
## alternative hypothesis: stationary
\end{verbatim}

\begin{Shaded}
\begin{Highlighting}[]
\DocumentationTok{\#\# normality: Jarque{-}Bera}
\FunctionTok{jarque.bera.test}\NormalTok{(xgrowth)}
\end{Highlighting}
\end{Shaded}

\begin{verbatim}
## 
##  Jarque Bera Test
## 
## data:  xgrowth
## X-squared = 2902.3, df = 2, p-value < 2.2e-16
\end{verbatim}

\begin{Shaded}
\begin{Highlighting}[]
\DocumentationTok{\#\# mean zero: z{-}test}
\NormalTok{sigmaLR }\OtherTok{\textless{}{-}} \FunctionTok{sum}\NormalTok{(}\FunctionTok{sapply}\NormalTok{(}\SpecialCharTok{{-}}\DecValTok{100}\SpecialCharTok{:}\DecValTok{100}\NormalTok{, gamma, }\AttributeTok{x =}\NormalTok{ xgrowth))}
\NormalTok{t\_stat  }\OtherTok{\textless{}{-}} \FunctionTok{mean}\NormalTok{(xgrowth)}\SpecialCharTok{/}\NormalTok{(}\FunctionTok{sqrt}\NormalTok{(sigmaLR}\SpecialCharTok{/}\NormalTok{t\_max))}
\NormalTok{p\_value }\OtherTok{\textless{}{-}}\NormalTok{ (}\DecValTok{1}\SpecialCharTok{{-}}\FunctionTok{pnorm}\NormalTok{(}\FunctionTok{abs}\NormalTok{(t\_stat)))}\SpecialCharTok{*}\DecValTok{2}

\CommentTok{\# t.test(xgrowth, mu = 0)}
\CommentTok{\# mean(xgrowth)/(sqrt((t\_max/(t\_max{-}1))*var(xgrowth)/t\_max)) \# for compatibility with t{-}test()}
\NormalTok{t\_stat\_unadjasted }\OtherTok{\textless{}{-}} \FunctionTok{mean}\NormalTok{(xgrowth)}\SpecialCharTok{/}\NormalTok{(}\FunctionTok{sqrt}\NormalTok{(}\FunctionTok{var}\NormalTok{(xgrowth)}\SpecialCharTok{/}\NormalTok{t\_max))}
\NormalTok{p\_value\_unadjasted }\OtherTok{\textless{}{-}}\NormalTok{ (}\DecValTok{1}\SpecialCharTok{{-}}\FunctionTok{pnorm}\NormalTok{(}\FunctionTok{abs}\NormalTok{(t\_stat\_unadjasted)))}\SpecialCharTok{*}\DecValTok{2}

\FunctionTok{rbind}\NormalTok{(}\AttributeTok{p\_value\_unadjasted =}\NormalTok{ p\_value\_unadjasted,}
      \AttributeTok{p\_value =}\NormalTok{ p\_value)}
\end{Highlighting}
\end{Shaded}

\begin{verbatim}
##                            [,1]
## p_value_unadjasted 0.000000e+00
## p_value            4.884981e-15
\end{verbatim}

\begin{Shaded}
\begin{Highlighting}[]
\CommentTok{\# We can study when adjustment really matters using the results of the previous}
\CommentTok{\# Monte Carlo simulation}
\NormalTok{ybar }\OtherTok{\textless{}{-}} \FunctionTok{colMeans}\NormalTok{(x\_list)}
\NormalTok{sigma }\OtherTok{\textless{}{-}} \FunctionTok{apply}\NormalTok{(x\_list, }\DecValTok{2}\NormalTok{, }\ControlFlowTok{function}\NormalTok{(x) }\FunctionTok{sqrt}\NormalTok{(}\FunctionTok{gamma}\NormalTok{(x, }\AttributeTok{k =} \DecValTok{0}\NormalTok{)}\SpecialCharTok{/}\FunctionTok{length}\NormalTok{(x)))}
\NormalTok{sigmaLR }\OtherTok{\textless{}{-}} \FunctionTok{apply}\NormalTok{(x\_list, }\DecValTok{2}\NormalTok{, }\ControlFlowTok{function}\NormalTok{(x) }\FunctionTok{sqrt}\NormalTok{(}\FunctionTok{sum}\NormalTok{(}\FunctionTok{sapply}\NormalTok{(}\SpecialCharTok{{-}}\DecValTok{3}\SpecialCharTok{:}\DecValTok{3}\NormalTok{, gamma, }\AttributeTok{x =}\NormalTok{ x))}\SpecialCharTok{/}\FunctionTok{length}\NormalTok{(x)))}
\NormalTok{t\_stat }\OtherTok{\textless{}{-}}\NormalTok{ ybar}\SpecialCharTok{/}\NormalTok{sigma}
\NormalTok{t\_stat\_adjusted }\OtherTok{\textless{}{-}}\NormalTok{ ybar}\SpecialCharTok{/}\NormalTok{sigmaLR}
\NormalTok{type1error     }\OtherTok{\textless{}{-}} \FunctionTok{mean}\NormalTok{(}\FunctionTok{abs}\NormalTok{(t\_stat) }\SpecialCharTok{\textgreater{}} \FunctionTok{qnorm}\NormalTok{(}\FloatTok{0.975}\NormalTok{))}
\NormalTok{type1error\_adj }\OtherTok{\textless{}{-}} \FunctionTok{mean}\NormalTok{(}\FunctionTok{abs}\NormalTok{(t\_stat\_adjusted) }\SpecialCharTok{\textgreater{}} \FunctionTok{qnorm}\NormalTok{(}\FloatTok{0.975}\NormalTok{))}
\end{Highlighting}
\end{Shaded}

\begin{Shaded}
\begin{Highlighting}[]
\CommentTok{\# distributions}
\FunctionTok{hist}\NormalTok{(t\_stat, }\AttributeTok{breaks =} \DecValTok{100}\NormalTok{, }\AttributeTok{freq =} \ConstantTok{FALSE}\NormalTok{, }\AttributeTok{col =} \StringTok{"tomato"}\NormalTok{, }
     \AttributeTok{ylim =} \FunctionTok{c}\NormalTok{(}\DecValTok{0}\NormalTok{, }\FloatTok{0.45}\NormalTok{), }\AttributeTok{xlim =} \FunctionTok{c}\NormalTok{(}\SpecialCharTok{{-}}\DecValTok{5}\NormalTok{, }\DecValTok{6}\NormalTok{), }\AttributeTok{xlab =} \StringTok{"t{-}stat"}\NormalTok{, }
     \AttributeTok{main =} \StringTok{"Distribution of the t{-}statistic in Monte Carlo simulation"}\NormalTok{)}
\FunctionTok{hist}\NormalTok{(t\_stat\_adjusted, }\AttributeTok{breaks =} \DecValTok{100}\NormalTok{, }\AttributeTok{freq =} \ConstantTok{FALSE}\NormalTok{, }\AttributeTok{add =} \ConstantTok{TRUE}\NormalTok{, }\AttributeTok{col =} \StringTok{"lightgreen"}\NormalTok{)}
\FunctionTok{lines}\NormalTok{(}\FunctionTok{density}\NormalTok{(}\FunctionTok{rnorm}\NormalTok{(}\FloatTok{1e6}\NormalTok{, }\AttributeTok{mean =} \DecValTok{0}\NormalTok{, }\AttributeTok{sd =} \DecValTok{1}\NormalTok{)), }\AttributeTok{lwd =} \DecValTok{4}\NormalTok{, }\AttributeTok{col =} \StringTok{"black"}\NormalTok{)}
\FunctionTok{legend}\NormalTok{(}\StringTok{"topright"}\NormalTok{, }\FunctionTok{c}\NormalTok{(}\StringTok{"Unadjusted"}\NormalTok{, }\StringTok{"Adjusted"}\NormalTok{), }\AttributeTok{col=}\FunctionTok{c}\NormalTok{(}\StringTok{"tomato"}\NormalTok{, }\StringTok{"lightgreen"}\NormalTok{), }\AttributeTok{lwd=}\DecValTok{6}\NormalTok{)}
\FunctionTok{abline}\NormalTok{(}\AttributeTok{v =} \FunctionTok{qnorm}\NormalTok{(}\FunctionTok{c}\NormalTok{(}\FloatTok{0.025}\NormalTok{, }\FloatTok{0.975}\NormalTok{)), }\AttributeTok{lty =} \DecValTok{2}\NormalTok{, }\AttributeTok{lwd =} \DecValTok{2}\NormalTok{)}
\FunctionTok{text}\NormalTok{(}\AttributeTok{x=}\FunctionTok{c}\NormalTok{(}\FloatTok{6.5}\NormalTok{, }\FloatTok{6.5}\NormalTok{), }\AttributeTok{y=}\FunctionTok{c}\NormalTok{(}\FloatTok{0.3}\NormalTok{, }\FloatTok{0.25}\NormalTok{), }
     \AttributeTok{labels=}\FunctionTok{c}\NormalTok{(}\FunctionTok{paste0}\NormalTok{(}\StringTok{"Type 1 error: "}\NormalTok{, }\FunctionTok{round}\NormalTok{(type1error}\SpecialCharTok{*}\DecValTok{100}\NormalTok{, }\DecValTok{1}\NormalTok{), }\StringTok{"\%"}\NormalTok{),}
              \FunctionTok{paste0}\NormalTok{(}\StringTok{"Type 1 error adjusted: "}\NormalTok{, }\FunctionTok{round}\NormalTok{(type1error\_adj}\SpecialCharTok{*}\DecValTok{100}\NormalTok{, }\DecValTok{1}\NormalTok{), }\StringTok{"\%"}\NormalTok{)), }
     \AttributeTok{col=}\FunctionTok{c}\NormalTok{(}\StringTok{"tomato"}\NormalTok{, }\StringTok{"lightgreen"}\NormalTok{), }\AttributeTok{pos =} \DecValTok{2}\NormalTok{)}
\end{Highlighting}
\end{Shaded}

\includegraphics{_main_files/figure-latex/unnamed-chunk-14-1.pdf}

\hypertarget{session-1-session1}{%
\chapter{Session 1 \{session1\}}\label{session-1-session1}}

\hypertarget{parts}{%
\chapter{Parts}\label{parts}}

You can add parts to organize one or more book chapters together. Parts can be inserted at the top of an .Rmd file, before the first-level chapter heading in that same file.

Add a numbered part: \texttt{\#\ (PART)\ Act\ one\ \{-\}} (followed by \texttt{\#\ A\ chapter})

Add an unnumbered part: \texttt{\#\ (PART\textbackslash{}*)\ Act\ one\ \{-\}} (followed by \texttt{\#\ A\ chapter})

Add an appendix as a special kind of un-numbered part: \texttt{\#\ (APPENDIX)\ Other\ stuff\ \{-\}} (followed by \texttt{\#\ A\ chapter}). Chapters in an appendix are prepended with letters instead of numbers.

\hypertarget{footnotes-and-citations}{%
\chapter{Footnotes and citations}\label{footnotes-and-citations}}

\hypertarget{footnotes}{%
\section{Footnotes}\label{footnotes}}

Footnotes are put inside the square brackets after a caret \texttt{\^{}{[}{]}}. Like this one \footnote{This is a footnote.}.

\hypertarget{citations}{%
\section{Citations}\label{citations}}

Reference items in your bibliography file(s) using \texttt{@key}.

For example, we are using the \textbf{bookdown} package \citep{R-bookdown} (check out the last code chunk in index.Rmd to see how this citation key was added) in this sample book, which was built on top of R Markdown and \textbf{knitr} \citep{xie2015} (this citation was added manually in an external file book.bib).
Note that the \texttt{.bib} files need to be listed in the index.Rmd with the YAML \texttt{bibliography} key.

The RStudio Visual Markdown Editor can also make it easier to insert citations: \url{https://rstudio.github.io/visual-markdown-editing/\#/citations}

\hypertarget{blocks}{%
\chapter{Blocks}\label{blocks}}

\hypertarget{equations}{%
\section{Equations}\label{equations}}

Here is an equation.

\begin{equation} 
  f\left(k\right) = \binom{n}{k} p^k\left(1-p\right)^{n-k}
  \label{eq:binom}
\end{equation}

You may refer to using \texttt{\textbackslash{}@ref(eq:binom)}, like see Equation \eqref{eq:binom}.

\hypertarget{theorems-and-proofs}{%
\section{Theorems and proofs}\label{theorems-and-proofs}}

Labeled theorems can be referenced in text using \texttt{\textbackslash{}@ref(thm:tri)}, for example, check out this smart theorem \ref{thm:tri}.

\begin{theorem}
\protect\hypertarget{thm:tri}{}\label{thm:tri}For a right triangle, if \(c\) denotes the \emph{length} of the hypotenuse
and \(a\) and \(b\) denote the lengths of the \textbf{other} two sides, we have
\[a^2 + b^2 = c^2\]
\end{theorem}

Read more here \url{https://bookdown.org/yihui/bookdown/markdown-extensions-by-bookdown.html}.

\hypertarget{callout-blocks}{%
\section{Callout blocks}\label{callout-blocks}}

The R Markdown Cookbook provides more help on how to use custom blocks to design your own callouts: \url{https://bookdown.org/yihui/rmarkdown-cookbook/custom-blocks.html}

\hypertarget{sharing-your-book}{%
\chapter{Sharing your book}\label{sharing-your-book}}

\hypertarget{publishing}{%
\section{Publishing}\label{publishing}}

HTML books can be published online, see: \url{https://bookdown.org/yihui/bookdown/publishing.html}

\hypertarget{pages}{%
\section{404 pages}\label{pages}}

By default, users will be directed to a 404 page if they try to access a webpage that cannot be found. If you'd like to customize your 404 page instead of using the default, you may add either a \texttt{\_404.Rmd} or \texttt{\_404.md} file to your project root and use code and/or Markdown syntax.

\hypertarget{metadata-for-sharing}{%
\section{Metadata for sharing}\label{metadata-for-sharing}}

Bookdown HTML books will provide HTML metadata for social sharing on platforms like Twitter, Facebook, and LinkedIn, using information you provide in the \texttt{index.Rmd} YAML. To setup, set the \texttt{url} for your book and the path to your \texttt{cover-image} file. Your book's \texttt{title} and \texttt{description} are also used.

This \texttt{gitbook} uses the same social sharing data across all chapters in your book- all links shared will look the same.

Specify your book's source repository on GitHub using the \texttt{edit} key under the configuration options in the \texttt{\_output.yml} file, which allows users to suggest an edit by linking to a chapter's source file.

Read more about the features of this output format here:

\url{https://pkgs.rstudio.com/bookdown/reference/gitbook.html}

Or use:

\begin{Shaded}
\begin{Highlighting}[]
\NormalTok{?bookdown}\SpecialCharTok{::}\NormalTok{gitbook}
\end{Highlighting}
\end{Shaded}


  \bibliography{book.bib,packages.bib}

\end{document}
